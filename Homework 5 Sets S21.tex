\documentclass{article}
\usepackage{amsmath}
\usepackage{amssymb}
\usepackage{enumerate}
\usepackage{graphicx}
\usepackage{geometry}
%\usepackage{fullpage}

\newcommand{\R}{\mathbb{R}}
\newcommand{\N}{\mathbb{N}}
\newcommand{\Z}{\mathbb{Z}}
\newcommand{\Q}{\mathbb{Q}}

\renewcommand{\iff}{\leftrightarrow}

\parindent 0in

\begin{document}

\noindent\textbf{Math 243 \hfill Homework 5: Sets}\\[0.0625in]
\mbox{}{\textbf{Due Friday, March 5, 2021}}\hfill\\

Please submit one set of well-written solutions per group.  Upload your completed solution set to Gradescope in PDF format. At the beginning of your document, include a brief statement of who contributed to the work.  Each solution should be written in paragraph form using complete sentences.  For each question, be sure to state any assumptions you have made and discuss your conclusions. 

\begin{enumerate} 

\item Consider the following sets:
$$A=\{n\in \mathbb{Z^+}:n \textrm{ is odd}\} $$
$$B=\{m\in \mathbb{Z^+}:m \textrm{ is prime}\} $$
$$C=\{4p+3:p\in \mathbb{Z^+}\} $$
$$D=\{x\in \mathbb{R}:x^2-8x+15=0\} $$
For each set in the list, discuss whether or not it is a subset of each of the other sets in the list, and why.  Draw a Venn diagram to illustrate the relationships among all four sets.

\item Let $\Sigma=\{r,s,t\}$ and let $\Sigma^*$ be the set of all possible finite strings that can be formed from the elements of $\Sigma$.  Repetition is allowed. The null string $\lambda$ is also a member of $\Sigma^*$.
\begin{enumerate}
	\item List the elements in the set $\{ w \in \Sigma ^* :\textrm{length}(w) \leq 2\}.$ How many elements are there?  (Remember to include the null string.)
	\item List the elements in the set $\{ w \in \Sigma ^* :\textrm{length}(w) \leq 3\}.$ How many elements are there? [Hint: try to organize the list of possibilities in some way to make it easy to see whether you have found them all.]
	\item How many elements are there in the set $\{ w \in \Sigma^*:\textrm{length}(w) \leq n\}?$  Find a formula in terms of $n$.  You can write your answer as a sum; if you have worked with sums like this before, you should also try to find a closed form solution. 
\end{enumerate}

\item \begin{enumerate}
	\item Make up an example from real life to illustrate a Cartesian product.  
	\item Make up an example from real life to illustrate a power set.  
	\item Make up an example from real life to illustrate a partition.
\end{enumerate}

Be sure to explain how your examples fulfill the necessary criteria for the thing they are illustrating.  Be creative; don't just use examples we have done in class.


\end{enumerate}

\end{document} 