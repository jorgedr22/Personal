\documentclass{article}
\usepackage{amsmath}
\usepackage{amssymb}
\usepackage{enumerate}
\usepackage{graphicx}
\usepackage{geometry}
\usepackage{enumitem}
\usepackage{xcolor}
\usepackage{tikz}
\usepackage{amsmath}
\usepackage{nicematrix}


\newcommand{\R}{\mathbb{R}}
\newcommand{\N}{\mathbb{N}}
\newcommand{\Z}{\mathbb{Z}}
\newcommand{\Q}{\mathbb{Q}}

\parindent 0in

\begin{document}

\noindent\textbf{Math 243 Discrete Mathematics \hfill Homework 6}

\begin{enumerate}
    \item In each question part below, I will list a set $\mathit{S}$ and 
    a rule that defines a relation $\mathit{R}$ on $\mathit{S}$ as follows: 
    $(\mathit{m,n}) \in R$ if $\mathit{m}$and $\mathit{n}$ satisfy the given rule. 
    For each set and rule, do the following five things: 
%qustion 1
\begin{enumerate}[label=\bfseries \roman*]
\item List the order pairs in the relation.
\item Draw the associated arrow diagram.
\item Give the matrix representation of the relation. 
Be sure to label the rows and columns with the elements of $S$.
\item Identify which properties the relation has: reflexive, 
antireflexive, or neither; symmetric, antisymmetric, or neither; 
transitive or not transitive. Be sure to explain your answers.
\item Decide whether the relation is an equivalence relation. 
If so, identify the equivalence classes.     
\end{enumerate}

\begin{enumerate}
    \item $\mathit{S}$ = \{4,9,17\}; $m \geq n$.    
    \item $\mathit{S}$ = \{0,2,5\}; $mn$ = 0.
    \item $\mathit{S}$ = \{1,2,6,7,11\}; $m \equiv n \pmod{3}$
\end{enumerate}
%answer to question 1
\underline{Answers for (a):}    

\begin{itemize}
    \item R = \{$(m,n):m \geq n$ and $m,n \in S$\}
    \\R = \{(17,4),(9,4),(17,9),(4,4),(9,9),(17,17)\}\\

    \item Diagram:\\

    \item Matrix:\\
        $\begin{pNiceMatrix}[first-row,first-col]
        & 4 & 9 & 17       \\
        4  & 1 & 0 & 0        \\ 
        9  & 1 & 1 & 0        \\
        17 & 1 & 1 & 1        \\
        \end{pNiceMatrix}$
        \hspace{1cm} Matrix = 
        $\begin{cases}
            1: (m,n)\in R\\
            0: otherwise\\  
        \end{cases}$ \\
    
    \item \begin{itemize}
       \item R is reflexive as $(m,n) \in R$ $\forall$ $m \in S$
       \item R is not anti reflexive. As there are elements in $S$ which are related to itself in R.
       \item R is not symmetric as (9,4) $\in R$ but (4,9) $\notin R$
       \item R is anti symmetric. As if(a,b) $\in R$ and (b,a) $\in R$ then a = b for: 
        \begin{itemize}
            \item[] a = b = 4 
            \item[] a = b = 9 
            \item[] a = b = 17
        \end{itemize}
        \item R is transitive. As if (a,b)$\in R$ and (b,c)$\in R$ then (a,c)$\in R$ holds. $\therefore a,b,c \in S$  
     \end{itemize}
    \item R is not an equivalence relation.
    
\end{itemize}

\pagebreak

\underline{Answers for (b):}    

\begin{itemize}
    \item R = \{$(m,n):mn = 0$ and $m,n \in S$\}
    \\R = \{(0,0),(0,2),(0,5),(2,0),(5,0)\}\\

    \item Diagram:\\

    \item Matrix:\\
        $\begin{pNiceMatrix}[first-row,first-col]
           & 0 & 2 & 5        \\
        0  & 1 & 1 & 1        \\ 
        2  & 1 & 0 & 0        \\
        5  & 1 & 0 & 0        \\
        \end{pNiceMatrix}$
        \hspace{1cm} Matrix = 
        $\begin{cases}
            1: mn = 0 \\
            0: otherwise\\  
        \end{cases}$ \\
    
    \item \begin{itemize}
       \item R is not reflexive as $(2,2) \notin R$ 
       \item R is symmetric as (0,2) $\in R$ and (2,0) $\in R$
        \item R is not transitive. Because (2,0) and (0,5) $\in R$ but (2,5) $\notin R$  
     \end{itemize}
    \item R is not an equivalence relation.
    
\end{itemize}

\underline{Answers for (c):}    

\begin{itemize}
    \item R = \{(m,n): $m \equiv n \pmod{3}$ and $m,n \in S$\}
    \\R = \{\}\\

    \item Diagram:\\

    \item Matrix:\\
        $\begin{pNiceMatrix}[first-row,first-col]
        & 4 & 9 & 17       \\
        4  & 1 & 0 & 0        \\ 
        9  & 1 & 1 & 0        \\
        17 & 1 & 1 & 1        \\
        \end{pNiceMatrix}$
        \hspace{1cm} Matrix = 
        $\begin{cases}
            1: (m,n)\in R\\
            0: otherwise\\  
        \end{cases}$ \\
    
    \item \begin{itemize}
       \item R is reflexive as $(m,n) \in R$ $\forall$ $m \in S$
       \item R is not anti reflexive. As there are elements in $S$ which are related to itself in R.
       \item R is not symmetric as (9,4) $\in R$ but (4,9) $\notin R$
       \item R is anti symmetric. As if(a,b) $\in R$ and (b,a) $\in R$ then a = b for: 
        \begin{itemize}
            \item[] a = b = 4 
            \item[] a = b = 9 
            \item[] a = b = 17
        \end{itemize}
        \item R is transitive. As if (a,b)$\in R$ and (b,c)$\in R$ then (a,c)$\in R$ holds. $\therefore a,b,c \in S$  
     \end{itemize}
    \item R is not an equivalence relation.
    
\end{itemize}
%question 2
\item Suppose $X$ is the set \{$a,b,c,d,e$\} and $P(X)$ is the power set of $X$. 
The domain of each of the relations below is $P(X)$. 
For each relation, describe the relation in words. Then determine whether it is 
reflexive, antireflexive, or neither; symmetric, anti-symmetric, or neither; and 
transitive or not transitive.

\begin{enumerate}
    \item $A$ is related to $B$ if $\lvert A-B \rvert$ = 1. The 
    vertical bars mean the cardinality of $A - B$.
    \item $A$ is related to $B$ if $A \subset B$. Note 
    that this notation means proper subset.
\end{enumerate}


%question 3
\item Determine whether the following relations are equivalence relations. 
If so, show that the relation fulfills the requirements for an equivalence relation, 
and describe the equivalence classes. If not, show how the relation fails to fulfill 
the requirements.

\begin{enumerate}
    \item The relation $\mathcal{C}$ is defined by $(x,y)\in \mathcal{C}$ iff cos($x$) = cos($y$), where $x,y \in R$.
    \item The relation $\varpropto$ on the domain $\Z^+\times\Z^+$ is defined by $(m,n) \varpropto (p,q)$ iff $mq = np$.
\end{enumerate}

\end{enumerate}

\end{document}