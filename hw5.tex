\documentclass{article}
\usepackage{amsmath}
\usepackage{amssymb}
\usepackage{enumerate}
\usepackage{graphicx}
\usepackage{geometry}
\usepackage{enumitem}
\usepackage{xcolor}
\graphicspath{ {/Users/jorgedelriocuriel/Downloads} }
\newcommand{\R}{\mathbb{R}}
\newcommand{\N}{\mathbb{N}}
\newcommand{\Z}{\mathbb{Z}}
\newcommand{\Q}{\mathbb{Q}}

\begin{document}

\noindent\textbf{Math 243 \hfill Homework 5 - Sets}

\textbf{We met up on Pi-Day (March 14, 2023) to celebrate 
the wonderful day of Pi and wanted to come together and 
complete the homework. Everyone met accordingly and 
contributed ideas and discussed thoughts on how to tackle 
the problems. Lucas contributed heavily on problem three, 
Matthew worked on problem one while Emma and Jorge wrote 
out every possible combination for problem two and figured 
out the formula that was asked in problem 2c. In short, 
everyone was putting in effort and we all ate pie to 
celebrate.}

\pagebreak

\underline{Problem 1}

\begin{enumerate}
    \item Consider the following sets:
$$A = \{n \in \Z^+ : \text{n is odd}\}$$
$$B = \{m\in\Z^+:\text{m is prime}\}$$
$$C = \{\text{4p+3 : }p\in\Z^+\}$$
$$D = \{x\in\R: x^2-8x+15=0 \}$$

For each set in the list, discuss whether or not it is a subset of 
each of the other sets in the list, and why. Draw a Venn diagram to 
illustrate the relationships among all four sets.

\includegraphics[scale=0.6]{vd}
\begin{itemize}
    \small
    \item $A$ is not a subset of $B$ because there exist odd integers 
    that are not prime (ie. 9, 15, etc).
    \item $A$ is not a subset of $C$ because not all odd integers are 
    of the form $4p + 3$.
\end{itemize}



\end{enumerate}

\end{document}